%%%%%%%%%%%%%%%%%%%%%%%%%%%%%%%%%%%%%%%%%%%%%%%%%%%
%
%  New template code for TAMU Theses and Dissertations starting Fall 2016.
%
%  Author: Sean Zachary Roberson
%	 Version 3.17.01
%  Last updated 1/10/2017
%
%%%%%%%%%%%%%%%%%%%%%%%%%%%%%%%%%%%%%%%%%%%%%%%%%%%
%%%%%%%%%%%%%%%%%%%%%%%%%%%%%%%%%%%%%%%%%%%%%%%%%%%%%%%%%%%%%%%%%%%%%
%%                           ABSTRACT
%%%%%%%%%%%%%%%%%%%%%%%%%%%%%%%%%%%%%%%%%%%%%%%%%%%%%%%%%%%%%%%%%%%%%

\chapter*{ABSTRACT}
\addcontentsline{toc}{chapter}{ABSTRACT} % Needs to be set to part, so the TOC doesnt add 'CHAPTER ' prefix in the TOC.

\pagestyle{plain} % No headers, just page numbers
\pagenumbering{roman} % Roman numerals
\setcounter{page}{2}

\indent This dissertation aims to investigate the underlying physics of surface electromagnetic waves that exist at a material interface, and have wavelength and phase velocity much smaller than those of a wave with same frequency but occurring in a homogeneous environment. By virtue of the small wavelength, the size of a resonating structure such as antenna and waveguide, can be miniaturized.

Recent advances in the semiconductor fabrication technology has resulted in emergence of two-dimensional (2D) materials engineered through epitaxially growing a stack of semiconducting layers. Such structures have enabled surface wave propagation, notably in the terahertz frequency range. An electromagnetic analysis using commercial software is particularly challenging for a multilayer structure of this kind, chiefly because of the presence of thin layers in the stack.

Integral equation (IE) based methods are ideally suited to solve for the electromagnetic fields of multilayer structures consisting of thin material layers, and to which an efficient and systematic formulation of Green functions (GFs) is of paramount importance. In this dissertation, a transmission-line network based approach is adopted to derive GFs for thin sheets in the spectral domain, and the associated spatial domain counterparts are computed through the Sommerfeld integrals (SIs).

The extraordinary electromagnetic properties of surface plasmons are demonstrated by a presentation of the properties of plasmonic antennas and of a super-resolution imaging scheme that is able to resolve objects separated only by a few nanometers.




We are living in an age where the evolution of semiconductor devices and components is contingent upon their miniaturization along with high-speed integration with the rest of the circuit. Unfortunately, electronics based systems will soon approach the theoretical speed and bandwidth limits. Photonic devices, in which broadband light pulses serve as information carriers, can at least overcome the bandwidth limitations. However, due to propagation of light in a dielectric medium, the size of the device cannot be reduced beyond the diffraction limit. A miniaturized design that can be integrated in to a circuit is therefore, not possible.





Plasmonics, which is a study of electromagnetic surface waves existing at a material interface, is a multidisciol


Plasmonic waves in solid-state are caused by collective oscillation of mobile charges inside or at the
surface of conductors. In particular, surface plasmonic waves propagating at the skin of metals have
recently attracted interest, as they reduce the wavelength of electromagnetic waves coupled to them by
up to ∼10 times, allowing one to create miniaturized wave devices at optical frequencies. In contrast,
plasmonic waves on two-dimensional (2D) conductors appear at much lower infrared and THz-GHz
frequencies, near or in the electronics regime, and can achieve far stronger wavelength reduction factor
reaching well above 100. In this thesis, we study the unique machinery of 2D plasmonic waves behind
this ultra-subwavelength confinement and explore how it can be used to create various interesting devices.

The performance, speed and ease-of-use of semiconductor devices, circuits and components is dependent on their miniaturization and integration into external devices. However, the integration of modern electronic devices for information processing and sensing is rapidly approaching its fundamental speed and bandwidth limitations, which is an increasingly serious problem that impedes further advances in many areas of modern science and technology. One of the most promising solutions is believed to be in replacing electronic signals (as information carriers) by light. However, a major problem with using electromagnetic waves as information carriers in optical signal-processing devices and integrated circuits is the low levels of integration and miniaturization available, which are far poorer than those achievable in modern electronics. This problem is a consequence of the diffraction limit of light in dielectric media, which does not allow the localization of electromagnetic waves into nanoscale regions much smaller than the wavelength of light in the material1.

The use of materials with negative dielectric permittivity is one of the most feasible ways of circumventing the diffraction limit and achieving localization of electromagnetic energy (at optical frequencies) into nanoscale regions as small as a few nanometres. The most readily available materials for this purpose are metals below the plasma frequency. Metal structures and interfaces are known to guide surface plasmon–polariton (SPP) modes2, electromagnetic waves coupled to collective oscillations of electron plasma in the metal. As a result, plasmonics is an area of nanophotonics beyond the diffraction limit that studies the propagation, localization and guidance of strongly localized SPP modes using metallic nanostructures. The recent rapid development of plasmonic waveguides whose mode confinement is not limited by the material parameters of the guiding structure has been primarily driven by the tantalizing prospect of combining the compactness of an electronic circuit with the bandwidth of a photonic network.





\pagebreak{}
