%%%%%%%%%%%%%%%%%%%%%%%%%%%%%%%%%%%%%%%%%%%%%%%%%%%
%
%  New template code for TAMU Theses and Dissertations starting Fall 2016.
%
%  Author: Sean Zachary Roberson
%	 Version 3.17.01
%  Last updated 1/10/2017
%
%%%%%%%%%%%%%%%%%%%%%%%%%%%%%%%%%%%%%%%%%%%%%%%%%%%
%%%%%%%%%%%%%%%%%%%%%%%%%%%%%%%%%%%%%%%%%%%%%%%%%%%%%%%%%%%%%%%%%%%%%
%%                           ABSTRACT
%%%%%%%%%%%%%%%%%%%%%%%%%%%%%%%%%%%%%%%%%%%%%%%%%%%%%%%%%%%%%%%%%%%%%

\chapter*{ABSTRACT}
\addcontentsline{toc}{chapter}{ABSTRACT} % Needs to be set to part, so the TOC doesnt add 'CHAPTER ' prefix in the TOC.

\pagestyle{plain} % No headers, just page numbers
\pagenumbering{roman} % Roman numerals
\setcounter{page}{2}

\indent This dissertation aims to investigate the underlying physics of surface electromagnetic waves that exist at a material interface, and have wavelength and phase velocity much smaller than those of a wave with same frequency but occurring in a homogeneous environment. By virtue of the small wavelength, the size of a resonating structure such as antenna and waveguide, can be miniaturized.

Recent advances in the semiconductor fabrication technology has resulted in emergence of two-dimensional (2D) materials engineered through epitaxially growing a stack of semiconducting layers. Such structures have enabled surface wave propagation, notably in the terahertz frequency range. An electromagnetic analysis using commercial software is particularly challenging for a multilayer structure of this kind, chiefly because of the presence of thin layers in the stack.

Integral equation (IE) based methods are ideally suited to solve for the electromagnetic fields of multilayer structures consisting of thin material layers, and to which an efficient and systematic formulation of Green functions (GFs) is of paramount importance. In this dissertation, a transmission-line network based approach is adopted to derive GFs for thin sheets in the spectral domain, and the associated spatial domain counterparts are computed through the Sommerfeld integrals (SIs).

The extraordinary electromagnetic properties of surface plasma waves are demonstrated by a presentation of super-resolution imaging scheme that has a capability to resolve objects separated only by a few nanometers.

\pagebreak{}
