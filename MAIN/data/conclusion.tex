\chapter{\uppercase {Conclusion and Future recommendations}}

In this work, plasmonic structures with an emphasis on antenna designs and imaging systems were studied that support subwavelength wave phenomena in the optical as well as the terahertz frequency domain. It was shown that the plasmonic antenna designs discussed herein provide an ideal outlet for realization of miniaturized communication devices. An in-depth theory dealing with the wave propagation mechanism in the form of surface plasmons was presented in the optical frequency domain. It was identified that the noble metals exhibit an optical frequency dielectric function with a negative real part, which is a necessary condition for the existence of surface plasmons. Likewise in the terahertz frequency region, plasmonic activity was observed in the few atoms wide electron channel of a high electron mobility transistor which is composed of an epitaxially grown semiconductor heterostructure.

A full wave analysis of plasmonic structures, and in particular semiconductor heterostructures, is inefficient using commercial software that are mostly based on differential equation discretization such as FEM and FDTD due to the presence of an extremely thin conducting layer. On the other hand, integral equation techniques in which the fields are computed by first constructing an appropriate Green function that is representative of the physical structure, and then followed by a method of moments discretization, are computationally efficient and most importantly, provide a great deal of insight on the wave mechanism in the structure. It must be stated that the mathematical formulation of integral equations is more involved and the integration routine must be cognizant of the singularities present in the integration kernels. In this work, an equivalent transmission line approach was followed to formulate the Green functions of an infinitesimally thin conductive sheet embedded in a semiconductor heterostructure. The complex plane integration was performed along the positive real line where the branch point singularities were circumvented by a triangular deformation of the integration path. Moreover, the mixed potentials formulation was adopted owing to comparatively weaker singular nature of the integral kernels. The results obtained by computing the  Sommerfeld integrals extracted from a magnetic vector potential formulation  for a free-standing conductive sheet show the existence of surface plasmons in the terahertz frequency regime.

The subwavelength nature of plasmonic structures was underlined by the dispersion relations and the resultant dispersion curves. Surface plasmons existing at a metal-dielectric interface at optical frequencies yield an analytical solutions of the dispersion relation. The dispersion relations for more complex semiconductor heterostructures were numerically solved in the terahertz frequency region using a robust complex-valued root-finding technique called the argument principle method. The results showed a much higher confinement of plasmons in the semiconductor heterostructure than the metal-dielectric interface, mainly due to the two-dimensional wave nature in the former case. In this regard, a super-resolution imaging scheme using a periodic structured illumination was proposed. It was shown that the terahertz standing plasma waves generated along the heterointerface, laterally enclosed by the transistor, can be used to resolve a sample with particle separation in the range of a few nanometers.

\section*{Recommendations for future work}
%
%
Terahertz plasmonics is currently being seen as the brightest prospect in terms of creating efficient terahertz devices that include sources and sensors. The thin plasma region inside a semiconductor heterostructure exhibits a resonant response in the terahertz frequency region that can be tuned using an active transistor environment. Unfortunately, at present practical efficiencies in terms of power can only be obtained at very low temperatures. Currently, most of the heterostructures are made from group III-V materials and their associated alloys. Extensive research is on-going to explored heterostructures that can operate at higher temperature. In this regard, materials such as perovskites and dichalcogenides have lately received an increased level of interest. Furthermore, the two-dimensional nature of the plasmonic structure is slowly evolving into a new research field termed as \emph{metasurfaces} \cite{Zhao_2011,Pors_2013}.

Various aspects of the analysis and design methods described throughout this dissertation be greatly improved. The surface conductivity of the 2DEG was assumed to be a scalar quantity. The quantum effects associated with a 2DEG due to external electric or magnetic fields can be incorporated in to the surface conductivity by modeling it as a tensor quantity.

The root-finding technique discussed in Chapter 4 currently involves a considerable amount of guess work in determining whether the poles are proper or improper. The routine can be improved by mapping the complex plane into a new coordinate system using a trigonometric transformation that essentially removes the branch points and its associated branch cuts.
