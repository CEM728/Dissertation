    \documentclass{article}
    \usepackage{verbatim}

    \usepackage{tikz}   %TikZ is required for this to work.  Make sure this exists before the next line

    \usepackage{tikz-3dplot} %requires 3dplot.sty to be in same directory, or in your LaTeX installation
    \usetikzlibrary{%
        decorations.pathreplacing,%
        decorations.pathmorphing%
    }
    \usepackage[active,tightpage]{preview}  %generates a tightly fitting border around the work
    \PreviewEnvironment{tikzpicture}
    \setlength\PreviewBorder{2mm}

    \begin{document}

    %Angle Definitions
    %-----------------

    %set the plot display orientation
    %syntax: \tdplotsetdisplay{\theta_d}{\phi_d}
    % \tdplotsetmaincoords{60}{00}

    %define polar coordinates for some vector
    %TODO: look into using 3d spherical coordinate system
    \pgfmathsetmacro{\rvec}{1}
    \pgfmathsetmacro{\thetavec}{60}
    \pgfmathsetmacro{\phivec}{60}

    %start tikz picture, and use the tdplot_main_coords style to implement the display 
    %coordinate transformation provided by 3dplot
    \begin{tikzpicture}[scale=2,tdplot_main_coords]

    %set up some coordinates 
    %-----------------------
    \coordinate (O) at (0,0,0);

    %set up some lengths 
    %-----------------------
    \pgfmathsetmacro{\len}{2}
    \pgfmathsetmacro{\h}{.5}
    \pgfmathsetmacro{\l}{.25}

    %determine a coordinate (P) using (r,\theta,\phi) coordinates.  This command
    %also determines (Pxy), (Pxz), and (Pyz): the xy-, xz-, and yz-projections
    %of the point (P).
    %syntax: \tdplotsetcoord{Coordinate name without parentheses}{r}{\theta}{\phi}
    \tdplotsetcoord{P}{\rvec}{\thetavec}{\phivec}

    %draw figure contents
    %--------------------

    \tikzset{facestyle/.style={fill=lightgray,draw=black,thin,dashed, line join=round}}
    % face "back" 
    \begin{scope}[canvas is zy plane at x=-\len]
      \path[facestyle,shade] (-.5,-\len) rectangle (0,\len);
    \end{scope}
    % face  "left"
    \begin{scope}[canvas is zx plane at y=-\len]
      \path[facestyle,shade] (-.5,-\len) rectangle (0,\len);
    \end{scope}
    % face "front"
    \begin{scope}[canvas is zy plane at x=\len]
      \path[facestyle, shade] (-.5,-\len) rectangle (0,\len);
    \end{scope}
    % face  "right"
    \begin{scope}[canvas is zx plane at y=\len]
      \path[facestyle, shade] (-.5,-\len) rectangle (0,\len);
    \end{scope}
    % face "top" 
    \begin{scope}[canvas is yx plane at z =0]
     \path[facestyle] (-\len,-\len) rectangle (\len,\len);
    \end{scope}

    %draw the main coordinate system axes
    \draw[thick,->] (0,0,0) -- (1,0,0) node[anchor=north east]{$x$};
    \draw[thick,->] (0,0,0) -- (0,1,0) node[anchor=north west]{$y$};
    \draw[thick,->] (0,0,0) -- (0,0,1) node[anchor=south]{$z$};

    %draw the negative coordinate system axes
    \draw[thick,dashed] (0,0,0) -- (-1,0,0);
    \draw[thick,dashed] (0,0,0) -- (0,-1,0);
    % \draw[thick,dashed] (0,0,0) -- (0,0,-.5);

% 
  % 
    % draw the Dipole
    \draw[-stealth,very thick, color=red] (0,0,\h - \l/2) -- (0,0,\h + \l/2) node[anchor=east, color = black]{$h$};
        % \draw[-stealth,very thick, color=red] (0,0,\h) -- (0,0,\h) node[anchor=south]{$h$};
    % 
    % 

    %draw a vector from origin to point (P) 
    \draw[-stealth,color=black] (O) -- (P) node[anchor=south]{$P$};

    % %draw projection on xy plane, and a connecting line
    % \draw[dashed, color=red] (O) -- (Pxy);
    % \draw[dashed, color=red] (P) -- (Pxy);

    %draw the angle \phi, and label it
    % %syntax: \tdplotdrawarc[coordinate frame, draw options]{center point}{r}{angle}{label options}{label}
    % \tdplotdrawarc{(O)}{0.2}{0}{\phivec}{anchor=north}{$\phi$}


    %set the rotated coordinate system so the x'-y' plane lies within the
    %"theta plane" of the main coordinate system
    %syntax: \tdplotsetthetaplanecoords{\phi}
    \tdplotsetthetaplanecoords{\phivec}

    %draw theta arc and label, using rotated coordinate system
    % \tdplotdrawarc[tdplot_rotated_coords]{(0,0,0)}{0.5}{0}{\thetavec}{anchor=south west}{$\theta$}

    %draw some dashed arcs, demonstrating direct arc drawing
    % \draw[dashed,tdplot_rotated_coords] (\rvec,0,0) arc (0:90:\rvec);
    % \draw[dashed] (\rvec,0,0) arc (0:90:\rvec);

    %set the rotated coordinate definition within display using a translation
    %coordinate and Euler angles in the "z(\alpha)y(\beta)z(\gamma)" euler rotation convention
    %syntax: \tdplotsetrotatedcoords{\alpha}{\beta}{\gamma}
    \tdplotsetrotatedcoords{\phivec}{\thetavec}{0}

    %translate the rotated coordinate system
    %syntax: \tdplotsetrotatedcoordsorigin{point}
    \tdplotsetrotatedcoordsorigin{(P)}


    \end{tikzpicture}

    \end{document}