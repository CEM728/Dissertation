        \documentclass{article}
        \usepackage{verbatim}

        \usepackage{tikz}   %TikZ is required for this to work.  Make sure this exists before the next line

        \usepackage{tikz-3dplot} %requires 3dplot.sty to be in same directory, or in your LaTeX installation
        \usetikzlibrary{%
            decorations.pathreplacing,%
            decorations.pathmorphing%
        }
        \usepackage[active,tightpage]{preview}  %generates a tightly fitting border around the work
        \PreviewEnvironment{tikzpicture}
        \setlength\PreviewBorder{0mm}

        \begin{document}

        %Angle Definitions
        %-----------------

        %set the plot display orientation
        %syntax: \tdplotsetdisplay{\theta_d}{\phi_d}
        \tdplotsetmaincoords{60}{130}

        %define polar coordinates for some vector
        %TODO: look into using 3d spherical coordinate system
        \pgfmathsetmacro{\rvec}{2}
        \pgfmathsetmacro{\thetavec}{70}
        \pgfmathsetmacro{\phivec}{60}

        %start tikz picture, and use the tdplot_main_coords style to implement the display 
        %coordinate transformation provided by 3dplot
        \begin{tikzpicture}[scale=2]



        %set up some lengths 
        %-----------------------
        \pgfmathsetmacro{\len}{2}
        \pgfmathsetmacro{\h}{.5}
        \pgfmathsetmacro{\l}{.25}

        %set up some coordinates 
        %-----------------------
        \coordinate (O) at (0,0,0);
        \coordinate (S) at (0,\h,0);

        % Define P through spherical coordinates
        \tdplotsetcoord{P}{\rvec}{\thetavec}{\phivec}
        %draw figure contents
        %--------------------

        \tikzset{facestyle/.style={fill=lightgray,draw=black,thin,dashed, line join=round}}
        % face "back" 
        \begin{scope}[canvas is yx plane at z=-\len]
          \path[facestyle,shade] (-.5,-\len) rectangle (0,\len);
        \end{scope}
        % face  "left"
        \begin{scope}[canvas is yz plane at x=-\len]
          \path[facestyle,shade] (-.5,-\len) rectangle (0,\len);
        \end{scope}
        % face "front"
        \begin{scope}[canvas is yx plane at z=\len]
          \path[facestyle, shade] (-.5,-\len) rectangle (0,\len) ;
        \end{scope}
        % face  "right"
        \begin{scope}[canvas is yz plane at x=\len]
          \path[facestyle, shade] (-.5,-\len) rectangle (0,\len);
        \end{scope}
        % face "top" 
        \begin{scope}[canvas is zx plane at y =0]
         \path[facestyle] (-\len,-\len) node[anchor=south east]{\textcircled{\raisebox{-0.9pt}{1}}} node[anchor=north west]{\textcircled{\raisebox{-0.9pt}{2}}} rectangle (\len,\len)  ;
        \end{scope}

        %draw the main coordinate system axes
        \draw[thin,-stealth] (0,0,0) -- (1,0,0) node[anchor=north east]{$x$};
        \draw[thin,-stealth] (0,0,0) -- (0,1,0) node[anchor=north east]{$z$};
        \draw[thin,-stealth] (0,0,0) -- (0,0,1) node[anchor=south]{$y$};

        %draw the negative coordinate system axes
        \draw[thin,dashed] (0,0,0) -- (-1,0,0);
        \draw[thin,dashed] (0,0,0) -- (0,0,-1);
        % \draw[thick,dashed] (0,0,0) -- (0,0,-.5);

        %draw a vector from origin to point (P) 
        \draw[-stealth,color=black] (O) -- node[anchor=south west]{$\bf{r}$} (P) ;

        %draw a vector from source to point (P) 
        \draw[-stealth,color=black] (S) -- node[anchor=south east]{$\bf{r'}$} (P);

                %draw a vector from source to point (P) 
        \draw[color=black] (P) node[anchor=south]{$P$};

        % draw the Dipole
        \draw[-stealth,very thick, color=red] (0,\h - \l/2,0) --  node[anchor=north east]{$Il$} (0,\h + \l/2,0) node[color = black, anchor=north east]{$h$};
            % \draw[-stealth,very thick, color=red] (0,0,\h) -- (0,0,\h) node[anchor=south]{$h$};


        \end{tikzpicture}

        \end{document}