\documentclass[12pt]{article}
% Horizontal Magnetic Dipole over a lossy half-space
\usepackage[utf8]{inputenc} % Use it to include other characters than ABC
\usepackage[T1]{fontenc}
\usepackage[cmex10]{amsmath}
\usepackage{calc}
% \usepackage{systeme} % For system of equations
\usepackage{amsfonts} % to load math symbols
\usepackage{mdwmath}
\usepackage{commath}
\usepackage{mdwtab}
\usepackage{hyperref}
\usepackage{physics} % For using the oridnary derivative nomenclature
\usepackage{datetime} % Insert date and time
\usepackage[letterpaper]{geometry}
\geometry{verbose,tmargin=1.25in,bmargin=1.25in,lmargin=1.4in,rmargin=1.15in}
\usepackage[nodisplayskipstretch,doublespacing]{setspace}
\setstretch{1.5}
\usepackage{tocloft}
% \usepackage[rm, tiny, center, compact]{titlesec}
\usepackage{indentfirst}
\usepackage{etoolbox}
% \AtBeginEnvironment{equation}{\leavevmode\singlespace}
% \AfterEndEnvironment{equation}{\endsinglespace\vskip0.5\baselineskip\noindent\ignorespaces}
% %
\usepackage{tocvsec2}
% \usepackage[titletoc]{appendix}
% \usepackage{appendix}
% \usepackage{tamuconfig}
%
% \usepackage{rotating}
\usepackage{graphicx}
\usepackage{pgfplots}
\usepackage{tikz}
\usepackage{standalone}
\usepackage[americanresistors,americaninductors]{circuitikz}
\usepackage{tikz-dimline} % For dimensional drawing
\usetikzlibrary{positioning}
\usetikzlibrary{arrows}
\usepackage{subfig}
% The following is done to hide ugly color boxes around the links
\usepackage{xcolor}
\hypersetup{
colorlinks,
linkcolor={red!50!black},
citecolor={blue!50!black},
urlcolor={blue!80!black}
}
% pdflatex -synctex=-1
\usepackage{mathptmx} % Times new Roman
\usepackage{times}
%
% ------------------------------- Useful Tricks Learnt
% Use ={}& to align subequations to the left

% Use = for single equations

% Use commath package to properly write differential operators and derivatives.

% Use \int\limits to nicely put integral limits

% To hide section numbers, place an asterisk after the section, e.g., \section*{}

% Put comments % in between the lines in order to avoid forming a new paragraph.

% To enter special characters into Inkspace figures, use Ctrl+U and then enter       the unicode. e.g., for \times symbol, the unicode is U+0D7. So the key entry would be Ctrl+U U+0d7 and then press enter.

% Put \eqref instead or \ref to reference equations. This will automatically put parantheses around the eq. number. amsmath package required.
%
% ----------------- To compile with references use the following order in Shell"
% 1. pdflatex filename.tex
% 2. bibtex filename (no extension)
% 3. bibtex filename (no extension)
% 4. pdflatex filename.tex
% -----------------

% Personal definitions
% Operators
\renewcommand{\v}[1]{\mathbf{#1}} % vectors
\newcommand{\ti}[1]{\tilde{#1}} % spectral representation

% Symbols
\renewcommand{\O}{\omega}  % omega
\newcommand{\E}{\varepsilon}  % epsilon
\renewcommand{\u}{\mu}  % mu
\newcommand{\p}{\rho}  % rho
\newcommand{\x}{\times}  % times
\renewcommand{\inf}{\infty}  % infinity
\newcommand{\infint}{\int\limits_{-\inf}^\inf} % integral by R
% \newcommand{\del}{\nabla}  % nabla operator
% \renewcommand{\^}{\hat}  % unit vector

\begin{document}
\title{\textsc{Equivalent Tranmission Line Models for Layered Structures with Sources}\\}
\date{\footnote{Last Modified: \currenttime, \today.}}
\maketitle
%
%
%
\section{Introduction}
%
In order to obtain accurate closed-form representations of the microstrip Green’s functions, it is often necessary to find the locations of the proper and improper surface-wave poles. In this paper, we present an efficient and robust iterative algorithm based on contraction mapping, which can locate all the proper and improper solutions of the characteristics equations of the grounded dielectric slab. The dielectric may also be lossy.

We seek to solve \eqref{eq:def} in the complex plane to obtain the zeros, $z_k$ in a given search region $\mathbb{C}$. For a multilayer problem, $f$ represents the dispersion relation which is the characteristic equation of the Transmission Line Green's Functions (TLGF). Depending on the location of the zeros of the function $f$ in the complex plane, $z_k$ correspond to the surface wave poles of the TLGF.
\begin{equation}
  f(z_k) = 0
  \label{eq:def}
\end{equation}
\subsection*{Counting the zeros}
In order to develop an efficient method of locating the zeros, we assume that the function $f$ is holomorphic, and non-zero at the boundary, $\Gamma$ of the region $\mathbb{C}$. Furthermore, the region is assumed rectangular. According to the Argument Principle Method (APM) initially proposed by Delves and Lyness, the number of zeros, $N$ inside a region with a boundary $\Gamma$ can be found by \cite{Delves1967c,Carpentier1982c,Gillan2006c}:
%
\begin{equation}
  N = \frac{1}{2 \pi} \oint \limits_{\Gamma} \dif {\{ \arg f(z)\} }
  \label{eq:numzeros}
\end{equation}
%
which states that each enclosed zero increments the argument by a factor of $2 \pi$. The integration around the contour is performed in a counter-clockwise manner and computed using an adaptive Gauss-Konrod quadrature (\emph{MATLAB}'s quadgk routine) with breakpoints defining the contour. In addition to \eqref{eq:numzeros}, the APM can also be stated in terms of the Cauchy's Integral Theorem \cite[pg. 71]{Krantz1999}:
%
\begin{equation}
  N = \frac{1}{2 \pi j} \oint \limits_{\Gamma} \frac{f'(z)}{f(z)} \dif z
  \label{eq:cauchyth}
\end{equation}
%
where the integrand is the logarithmic derivative of the function. Obtaining an analytical derivative of functions is cumbersome, specially in case of multilayer problems. Therefore, we approximate the derivative with a finite difference formula:
%
\begin{equation}
  f'(z) = \frac{f(z + h) - f(z - h)}{2 h}
  \label{eq:FD}
\end{equation}
%
where $h \sim \sqrt{\epsilon_m}z$ with $\epsilon_m = 2.2204 \times 10^{-16}$ as the double-precison machine accuracy \cite[pg. 230]{press2007numerical}.
%
\subsection*{Locating the zeros}
%
Aftering predicting the number of zeros through \eqref{eq:numzeros}, the next step involves the approximation of the dispersion function $f$ with an associated \emph{formal orthogonal polynomial} (FOP), $\mathcal P$ of degree $N$ with the condition that it has same roots, $z_k$ where $k = 1,2,...,N$ as $f$ \cite{Delves1967c,Kravanja1999}. The Lagrange representation of the polynomial is:
%
\begin{equation}
  \mathcal{P}_N(z) = \prod \limits_{k = 1}^N \left(z - z_k \right).
  \label{eq:poly}
\end{equation}
%
For orthogonality, we require the inner product,
%
\begin{equation}
  \langle z^k{,} \mathcal{P}_N(z)\rangle = \frac{1}{2 \pi j} \oint \limits_{\Gamma} z^k \mathcal{P}_N(z) \frac{f'(z)}{f(z)} \dif z, \quad \mathrm{with}\quad k = 0,1,...,N-1
  \label{eq:fop}
\end{equation}
%
to be zero \cite{Kravanja2000c}. The polynomial approximation of the original function reduces the complexity of the problem as techniques for finding polynomial roots are robust and well-established. Before we proceed, for the sake of convenience in obtaining the required polynomial, we express $\mathcal{P}$ in an alternate summation notation:
%
\begin{equation}
  \mathcal{P}_N(z) = \sum \limits_{k = 0}^N \alpha_k z^k
  \label{eq:poly_sum}
\end{equation}
%
For a \emph{monic polynomial}, $\alpha_N = 1$, and $\alpha_k$ are the sums of products of zeros, $Z_k$.

Next we look at a sequence of integrals:
%
\begin{equation}
  s_k = \frac{1}{2 \pi j} \oint \limits_{\Gamma} z^k \frac{f'(z)}{f(z)} \dif z, \quad \mathrm{with}\quad k = 0,1,2,...
  \label{eq:sk}
\end{equation}
%
where the $s_k$'s are the \emph{Newton moments} and related to the unknown coefficients $\alpha_k$'s through the \emph{Newton identities}:
%
\begin{equation}
  \begin{aligned}
    s_1 + \alpha_1 ={}& 0 \\
    s_2 + s_1 \alpha_1 + 2 \alpha_2 ={}& 0 \\
    {\vdots}\\
    s_N + s_{N-1} \alpha_{1} + ... + s_1 \alpha_{N-1} + N \alpha_N ={}& 0
    \label{eq:newtid}
  \end{aligned}
\end{equation}
%
After using the Newton's identities to construct the associated polynomial, we set up an eigenvalue problem in terms of two Hankel matrices, $\mathbf H$ and $\mathbf H^<$ which is derived from the Companion matrix of the polynomial \cite{Gillan2006c}:
%
\begin{equation}
  \mathbf H =
  \begin{bmatrix}
    s_1 & s_2 & \cdots & s_k \\
    s_2 & s_3 & \cdots & s_{k+1} \\
    s_3 & s_4 & \cdots & s_{k+2} \\
    \vdots & \vdots & \ddots & \vdots \\
    s_k & s_{k+1} & \cdots & s_{2k} \\
  \end{bmatrix}
  \label{eq:Hmat}
\end{equation}
%
and
%
\begin{equation}
  \mathbf H^< =
  \begin{bmatrix}
    s_0 & s_1 & \cdots & s_{k-1} \\
    s_1 & s_2 & \cdots & s_{k} \\
    s_2 & s_3 & \cdots & s_{k+1} \\
    \vdots & \vdots & \ddots & \vdots \\
    s_{k-1} & s_{k} & \cdots & s_{2k-2} \\
  \end{bmatrix}
  \label{eq:Hmat<}
\end{equation}
%
The roots of the polynomial $\mathcal P$ are the generalized eigenvalues, $\lambda$ of the matrix pencil:
%
\begin{equation}
  \left( \mathbf H - \lambda \mathbf H^< \right) \mathbf x = 0
  \label{eq:evp}
\end{equation}
%
where the column vector $\mathbf{x}$ is the eigenvector.
%
\subsection*{Refining the roots}
%
Approximating a function in a given contour with many zeros requires a higher-order polynomial that introduces computational problems. In addition, the integrals of the moments in \eqref{eq:sk} need to be evaluated with a higher-accuracy and the mapping between $s_k$ and $\alpha_k$ \eqref{eq:newtid} rseults in an ill-conditioned system. To overcome such pitfalls, a limit is enforced on the number of zeros in a given region. If the number of zeros exceeds a predetermined value $M$, the size of the search region is subdivided \cite{Delves1967c}. For problems pertinents to multilayer structures, a safe choice of $M$ is $3$.

The accuracy of the roots obtained from the eigenvalues,$\lambda_k$ of \eqref{eq:evp} is not always high. However, $\lambda_k$'s is an excellent inintial guess for any iterative root-search routine from the class of Householder's methods. We choose the \emph{Halley's} method having cubic convergence and the iteration formula:
%
\begin{equation}
  x_{n+1} = x_n - \frac {2 f(x_n) f'(x_n)} {2 {[f'(x_n)]}^2 - f(x_n) f''(x_n)}
  \label{eq:halley}
\end{equation}
%
with $f'(x)$ and $f''(x)$, the first and second order derivatives approximated by finite differences. In general the roots, $z_k$'s lie in the complex plane. The iteration \eqref{eq:halley} needs to be performed on both the real and imaginary parts simultaneously.

\subsection*{Branch Cuts}

The TLGF for a multilayer possesses two types singularities in the complex plane. We are primarily interested in finding the pole singularities that correspond to guided modes. In addition, the branch point singularities correspond to the radiation modes. Branch points arise due to the propagation constant, $k_{zi}$ of the layers having semi-infinite dimensions. The dispersion relation shows an even dependence on $k_{zi}$ of the layers that have finite width, hence there is no corresponding branch point.   \cite[Section~5.3a]{felsen1994radiation}. In order to satisfy Sommerfeld radiation condition that all fields must decay to zero at infinity, the proper sheet of the Riemann surface of the square-root function needs to be selected. This is accomplished by enforcing that the imaginary part of $k_{zi}$ that contributes a branch point, is negative everywhere.
%
\begin{equation}
  \Im (k_{zi}) < 0
  \label{eq:proper}
\end{equation}
%








%% Make this end of each document
\clearpage % Force Bibliography to the end of document
\bibliography{citations}
\bibliographystyle{ieeetr}

\end{document}
