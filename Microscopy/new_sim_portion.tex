\documentclass[11pt]{article}

% Insert style guide
\usepackage{my_thesis}
% \usepackage[style=ieee]{biblatex}
% \addbibresource{zubairy.bib}
% Specifiy the location of images to be used
\graphicspath{{figures/}}

\begin{document}
\title{\textsc{Plasma based Structured Illumination Microscopy}}
\date{\footnote{Last Modified: \currenttime, \today.}}
\maketitle


\section{Abstract}
%
We propose a high-resolution nanoscopy technique using two-dimensional plasma waves generated in a semiconductor heterostructure. The working principle is similar to structured illumination microscopy augmented by the high spatial frequency ans tunability of plasmons yielding resolution up to two orders of magnitude beyond the diffraction limit. The nature of the technique is linear, meaning a weak illumination signal is required, minimizing the chances of radiation damage of sample.
We present a linear high-resolution imaging scheme based on the plasma waves originating in the channel of field-effect transistors. The extremely small plasmonic wavelength along with a tunable illumination pattern in the far infrared region can resolve nano-scale objects over a broad range of frequencies.

In conventional wide-field fluorescent microscopy, a sample to be imaged is uniformly illuminated by light and the subsequent fluorescence is observed in the far-field through the objective of the microscope. The uniform nature of the illumination fundamentally restricts the resolution of the system to half the wavelength of light due to Abbe diffraction limit. With ever growing need to image tiny objects particularly in life sciences, modern microscopy techniques such as confocal and linear structured illumination microscopy use spatially non-uniform sources to illuminate the sample, resulting in resolution extending beyond the diffraction limit by a factor of $2$ \cite{Minsky1988, Gustafsson2005}. In confocal microscopy, a small portion of sample is illuminated by a focused beam generated through a pinhole. The beam is laterally shifted to completely scan the sample, creating a sequence of images. Each image passes through another pinhole on the detector side. A high resolution image of the sample is generated by processing the image sequence, however confocal microscopy is a slow imaging technique. Moreover, part of light is discarded by the pinhole which may leave the signal strength from weakly fluorescent samples undetectably low. Structured Illumination microscopy (SIM) is a wide-field technique in which a fine illumination pattern such as a sinusoidal standing wave is used to generate \emph{Moiré fringes} in the observed image. The high frequency content is mathematically reconstructed from a series of images acquired by shifting the pattern, yielding a high resolution image.
%%%%%%%%%%%%%%%
%%%%%%%%%%%%%%%
%%%%%%%%%%%%%%%
%%%%%%%%%%%%%%%
%%%%%%%%%%%%%%%
%%%%%%%%%%%%%%%
%%%%%%%%%%%%%%%
%%%%%%%%%%%%%%%
%%%%%%%%%%%%%%%
\section{Principle of Structured Illumination Microscopy}
%
Consider $I(\v r)$ as the sinusoidal illumination intensity:
%
\begin{equation}
  I(\v r) = 1 + \cos(\v k_{\p} \cdot \v r + \phi)
  \label{eq:intensity}
\end{equation}
where $\v k_{\p} = k_x \v{\^{x}} + k_y \v{\^{y}}$ is the spatial frequency wavevector,  $\v r = x \v{\^{x}} +  y \v{\^{y}}$ is the two-dimensional positional vector and $\phi$ is the pattern phase. The image of a sample $F(\v r)$ observed through a microscope can be expressed as:
%
\begin{equation}
  M(\v r) = \left[ F(\v r) \cdot I(\v r) \right] \otimes H(\v r)
  \label{eq:m_spatial_image}
\end{equation}
%
where $H(\v r)$ is the point spread function (PSF) of the microscope, and $\cdot, \otimes$ denote multiplication and convolution operations in the spatial domain respectively. A frequency domain representation of the image by taking the Fourier transform is expressed as:
%
\begin{equation}
  \begin{split}
    \ti M(\v k) &= \left[ \ti F(\v k) \otimes \ti I(\v k) \right] \cdot \ti H(\v k) \\
     &= \frac{1}{2} \left[ 2\ti F(\v k) + \ti F(\v k - \v k_{\p}) \e^{- \j \phi} + \ti F(\v k + \v k_{\p}) \e^{\j \phi} \right] \cdot \ti H(\v k)
  \end{split}
  \label{eq:m_ft}
\end{equation}

where $\sim$ over the letters indicates a frequency domain term and $\ti H(k)$ is the optical transfer function (OTF) of the microscope. As evident in \eqref{eq:m_ft}, a sinusoidal illumination pattern has three frequency components which generates an image which is linear combination of the sample along with two shifted versions as shown in Fig. \ref{fig:sim}(b). To reconstruct the sample, three different images need to be captured with different phase term $\phi$. The process can be expressed as a system of linear equations,
%
\begin{equation}
  \ti H(\v k) \cdot
  \begin{bmatrix}
    \ti F(\v k) \\
    \ti F(\v k - \v k_{\p}) \\
    \ti F(\v k + \v k_{\p})
  \end{bmatrix}
  =
  \begin{bmatrix}
    2 & \e^{-\j \phi_1} & \e^{\j \phi_1} \\
    2 & \e^{-\j \phi_2} & \e^{\j \phi_2} \\
    2 & \e^{-\j \phi_3} & \e^{\j \phi_3} \\
  \end{bmatrix}^{-1}
  \begin{bmatrix}
   \ti M_1(\v k) \\
   \ti M_2(\v k) \\
   \ti M_3(\v k)
  \end{bmatrix}
  \label{eq:reconstruction_algo}
\end{equation}
%
\begin{figure}[t!]
  \def\svgwidth{\linewidth}
  \input{figures/sim.pdf_tex}
  \caption{Resolution enhancement through SIM: (a) Diffraction limited observable region in frequency domain.  Moiré effect using a sinusoidal illumination pattern bringing high frequency content under the observable region. The sample is rotated: (b) $0 \degree$, (c) $60 \degree$, (d) $120 \degree$. (e) Doubling of lateral resolution with effective coverage area twice the size of (a)}
  \label{fig:sim}
\end{figure}
%
The phase shifts in \eqref{eq:reconstruction_algo} are known beforehand. Frequency content of the sample up to $k_{\p}$ can, therefore be observed due the Moiré effect which transports the high frequency information in to the obervation region. To achieve two-dimensional enhancement in resolution, the sample has to be rotated about the optical axis of the microscope or the angular distribution of the illumination needs to be varied.
%%%%%%%%%%%%%%%
%%%%%%%%%%%%%%%
%%%%%%%%%%%%%%%
%%%%%%%%%%%%%%%
%%%%%%%%%%%%%%%
%%%%%%%%%%%%%%%
%%%%%%%%%%%%%%%
%%%%%%%%%%%%%%%
%%%%%%%%%%%%%%%
\section{Generating Illumination pattern}
%
When a current is passed through a two-dimensional electron gas Possible sources of terahertz (THz) radiation have been investigated for several years now. For example, by epitaxially growing layers of different III–V semiconductors, THz quantum-cascade lasers, usually with a fixed frequency, have
been fabricated.1 The existence of instability in an electrically driven plasma demonstrates new approaches for efficient energy conversion as well as for designing and fabricating tunable THz emitters. Initially, the instability of a
two-dimensional (2D) plasma above a conducting half-space was addressed by Krasheninnikov and Chaplik.2 However, we show that when a layer lies above a thick conductor, it is incorrect to simply replace one of the frequencies in a
two-layer plasma dispersion equation by the surface plasmon frequency of the underlying substrate, as suggested by Krasheninnikov and Chaplik.2 It was later predicted by Kempa et al.3 (see also Ref. 4) that when a current is passed
through a stationary two-component 2D electron gas (2DEG), the Doppler shift in response frequency leads to a spontaneous generation of plasmon excitations without a wavenumber cutoff and subsequent Cherenkov radiation5 at sufficiently high carrier drift velocities
%
\begin{equation}
  \O_p^2 = \frac{N_s e^2 k}{m^{\ast}\E}\left(1 + \frac{\E - 1}{\E + 1}\e^{-2k  k d}\right)
  \label{eq:plasma_f_HEMT}
\end{equation}
%














%%%%%%%%%%%%%%%
%%%%%%%%%%%%%%%
%%%%%%%%%%%%%%%
\clearpage % Force Bibliography to the end of document on a new page.
% \printbibliography
% \addbibresource{zubairy}
\bibliography{zubairy}
\bibliographystyle{ieeetran}

\end{document}
