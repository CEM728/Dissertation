\documentclass[conference, 10pt]{IEEEtran}


%
\usepackage[T1]{fontenc} % optional
\usepackage[cmex10]{amsmath}
\usepackage{calc}
\usepackage{amsfonts} % to load math symbols
\usepackage{mdwmath}
\usepackage{commath}
\usepackage{physics} % For using the oridnary derivative nomenclature
\usepackage{mdwmath}
\usepackage{mdwtab}
\hyphenation{op-tical net-works semi-conduc-tor}


\usepackage{graphicx}
\usepackage{color}
\usepackage{placeins}
\usepackage{float}
\usepackage{hyperref}
% % The following is done to hide ugly color boxes around the links
\usepackage{xcolor}
\hypersetup{
colorlinks,
linkcolor={red!50!black},
citecolor={blue!50!black},
urlcolor={blue!80!black}
}


\usepackage{booktabs}
\usepackage{standalone}
\usepackage{filecontents}

\usepackage{tabularx,colortbl}
\usepackage{pgfplots}
\usepackage{tikz}
\usepackage[americanresistors,americaninductors]{circuitikz}
\usepackage{tikz-dimline} % For dimensional drawing
\tikzset{every picture/.append style={font=\normalsize}}
\usepackage{relsize}

\tikzset{fontscale/.style = {font=\relsize{#1}}
}
\usetikzlibrary{positioning}
\usetikzlibrary{arrows}
\usetikzlibrary{patterns}
\pgfplotsset{compat=newest}
%% the following commands are sometimes needed
\usetikzlibrary{plotmarks}
\usepackage{grffile}
\usepackage{mathtools,amssymb,lipsum}



%%%%%%%%%%
%%%%%%%%%% TIPS and TRICKS
%%%%%%%%%%
%
% ------------------------------- Useful Tricks Learnt
% Use ={}& to align subequations to the left

% Use = for single equations

% Use &= for split equations

% Use commath package to properly write differential operators and derivatives.

% Use \int\limits to nicely put integral limits

% For long equations, use align environment with \notag\\ as a linebreak.

% To hide section numbers, place an asterisk after the section, e.g., \section*{}

% Put comments % in between the lines in order to avoid forming a new paragraph.

% To enter special characters into Inkspace figures, use Ctrl+U and then enter       the unicode. e.g., for \times symbol, the unicode is U+0D7. So the key entry would be Ctrl+U U+0d7 and then press enter.

% Put \eqref instead or \ref to reference equations. This will automatically put parantheses around the eq. number. amsmath package required.
%
% ----------------- To compile with references use the following order in Shell"
% 1. pdflatex filename.tex
% 2. bibtex filename (no extension)
% 3. bibtex filename (no extension)
% 4. pdflatex filename.tex
% -----------------

% Personal definitions
% Operators
\renewcommand{\v}[1]{\mathbf{#1}} % vectors
\newcommand{\ti}[1]{\tilde{#1}} % spectral representation

% Symbols
\renewcommand{\O}{\omega}  % omega
\newcommand{\E}{\varepsilon}  % epsilon
\renewcommand{\u}{\mu}  % mu
\newcommand{\p}{\rho}  % rho
\newcommand{\x}{\times}  % times
\renewcommand{\inf}{\infty}  % infinity
\newcommand{\infint}{\int\limits_{-\inf}^\inf} % integral by R
\renewcommand{\del}{\nabla}  % nabla operator
\renewcommand{\^}{\hat}  % unit vector
% \newcommand*\diff{\mathop{}\!\mathrm{d}} % Define differential operator







\begin{document}

\title{An Integral Equation Scheme for Plasma based Thin Sheets}


% author names and affiliations
% use a multiple column layout for up to three different
% affiliations
\author{\IEEEauthorblockN{Hasan T. Abbas}
\IEEEauthorblockA{Department of Electrical and\\
Computer Engineering\\
Texas A\&M University\\
College Station, TX 77843-3128\\
Email: hasantahir@tamu.edu}
\and
\IEEEauthorblockN{Robert D. Nevels}
\IEEEauthorblockA{Department of Electrical and\\
Computer Engineering\\
Texas A\&M University\\
College Station, TX 77843-3128\\
Email: nevels@ece.tamu.edu}
}
%%%%%%%%%%%%%%%
%%%%%%%%%%%%%%%%
%%%%%%%%%%%%%%%%
%%%%%%%%%%%%%%%%
% make the title area
\maketitle


%
\begin{abstract}
  %\boldmath
  An integral equation formulation for a thin dielectric sheet is presented using the surface equivalence theorem. The advantageous properties of plasma waves, chief among them surface wave propagation are briefly discussed. Numerical results are presented to illustrate the scattering properties of the sheet with different material properties.
\end{abstract}

\IEEEpeerreviewmaketitle
%%%%%%%%%%%%%%%
%%%%%%%%%%%%%%%%
%%%%%%%%%%%%%%%%
%%%%%%%%%%%%%%%%
\section{Introduction}

The emergence of high-precision nanoscale fabrication techniques has led to an increased interest in two-dimensional (2D) materials and electronic systems of late, especially in the terahertz frequency regime. One particular intriguing example is the two-dimensional electron gas (2DEG) existing in the multilayer stack of semiconductor structures like high-electron mobility transistors (HEMTs), with remarkable electrical properties such as high conductivity with high values of free-carrier densities. The 2DEG is extremely thin as compared to other layer thicknesses in the stack and therefore, its scattering properties can be found by modeling it as a two-dimensional plasma. An interaction between an external electromagnetic radiation and plasma results in 2D plasmons (surface waves). In this paper, we formulate the scattering response of an infinitesimally thin flat layer of plasma surrounded by free-space using the surface equivalence theorem.
%%%%%%%%%%%%%%%
%%%%%%%%%%%%%%%%
%%%%%%%%%%%%%%%%
%%%%%%%%%%%%%%%%
\section{Theory}

\subsection{Surface Plasmons}
%
The electrical properties of any material can be characterized by a frequency-dependent permittivity:
%
\begin{equation}
  \E(\O)=\E_r - j\frac{\sigma(\O)}{\omega \E_0}
  \label{eq:epsilon}
\end{equation}
%
where $\E_r$ is the permittivity of the material at dc frequency and $\sigma$ is the conductivity given by a Drude-type model \cite{burke2000high}:
%
\begin{equation}
  \sigma(\O) = \frac{N e^2 \tau}{m^{\ast}}\frac{1}{1 + j \O \tau}
  \label{eq:conductivity}
\end{equation}
%
The parameters $e$ and $m^*$ are the charge and effective mass of electron respectively, $N$ is free-charge density, and $\tau$ is the scattering time of free charges in the 2DEG and can determined with the physical quantity, mobility $\mu_e$ can be computed from:
\begin{equation}
  \tau  = \frac{m^{\ast} \mu_e}{e}.
  \label{eq:tau}
\end{equation}
%
The dispersion relation of the 2D plasma waves can be written in terms plasma frequency $\O_p$ and the wave-number $k$:
%
\begin{equation}
  \O_{p} =  \sqrt{\frac{2 \pi e^2 N} {m^{\ast}} k}.
  \label{eq:N_2d}
\end{equation}
%
\subsection{Surface Integral Equation}

Consider a flat plasma sheet of length $L$ and thickness $t$ excited by a $\mathrm{TM_z}$ polarized plane wave as illustrated in Fig. \ref{fig:plate}. The plasma is assumed nonmagnetic and the dielectric constant is determined from \eqref{eq:epsilon}-\eqref{eq:N_2d}. By applying the surface equivalence theorem \cite[p. 328-333]{balanis2012advanced}, the plasma sheet is replaced by an equivalent set of surface currents. For the case of 2DEG plasma, the thickness is treated in its limiting case of $t \to 0$. The resulting electric field integral equation (EFIE) in a homogeneous free-space is written as:
%
\begin{equation}
  E_i = \frac{\O \u}{4} \int \limits_{0}^{L} J_z(x') \left[ H_0^{(2)}(k_1 |x - x'|) + H_0^{(2)}(k_2 |x - x'|)\right] \mathrm{d}x'
  \label{eq:plate}
\end{equation}
%
where $\mu$ is the free-space permeability, $J_z$ is the yet-unknown surface electric current, $H_0^{(2)}(\cdot)$ is the zero-order Hankel function of the second kind and $k_i$ with $i = 1,2$ being the corresponding wave-numbers of the free-space and plasma respectively.
%
% \begin{figure}[h]
%     \normalsize
%   \begin{center}
%     \noindent
%     \includegraphics[width=3in]{figures/tm_plate.tex}
%     \caption{Thin Plasma sheet with $\mathrm{TM_z}$ polarization}
%     \label{fig:plate}
%   \end{center}
% \end{figure}
%
\begin{figure}[h]
  \normalsize
  \centering
  \includestandalone[width=.5\textwidth]{figures/tm_plate}
  \caption{Thin Plasma sheet with $\mathrm{TM_z}$ polarization}
  \label{fig:plate}
\end{figure}
%
%%%%%%%%%%%%%%%
%%%%%%%%%%%%%%%%
%%%%%%%%%%%%%%%%
%%%%%%%%%%%%%%%%
\section{Numerical Results}
%
A method of moments (MoM) solution to compute the current $J_z$ in \eqref{eq:plate} is implemented using pulse basis functions with point matching method. The resulting integrals contain a logarithmic singularity due to self terms ($ x = x'$), however, it can be easily removed through a simple step invovling integration by parts followed by using recurrence property of Hankel functions \cite[p. 361]{amari1995efficient,abramowitz1968handbook}, resulting in a well-behaved integral. Assuming $x = 0$, we obtain:
%
\begin{align}
  Z_{self} = \frac{\O \u}{4} \Delta \left[H_0^{(2)}(k_1 \Delta) +  H_0^{(2)}(k_2 \Delta)\right] + \notag\\
  \frac{\O \u}{4} \int \limits_{0}^{\Delta} x'\left[ k_1 H_1^{(2)}(k_1 x')  + k_2 H_1^{(2)}(k_2 x')\right] \mathrm{d}x'
  \label{eq:self}
\end{align}
%
where $\Delta$ is the width of the pulse basis functions.
%
\subsection{Current Distribution}
%
Fig. \ref{fig:edgeon} shows the absolute value of the tangential surface electric current on a TM-wave excited plate of length $3 \lambda$ at edge-on ($\phi_i = \pi$). Gallium Arsenide ($\mathrm{GaAs}/\mathrm{AlGaAs}$) and Strontium Titanate ($\mathrm{LaAlO_3}/\mathrm{SrTiO_3}$) based 2DEG plasma sheets are considered where material data has been taken from measurements in \cite{burke2000high} and \cite{herranz2012high} respectively, and the results are compared with a PEC plate of same length \cite{senior1979backscattering}.

% Fig. \ref{fig:normal} shows the current distribution under normal incidence $(\phi_i = \pi/2)$.
%
\begin{figure}[h]
  \begin{center}
    \noindent
    \includegraphics[width=3in]{figures/currents_edgeon_3.tex}
    \caption{Current Distributions on a $3\lambda$ plate at edge-on incidence}
    \label{fig:edgeon}
  \end{center}
\end{figure}
%
% \begin{figure}[h]
%   \normalsize
%   \centering
%   \includestandalone[width=.5\textwidth]{figures/currents_edgeon_3}
%   \caption{Current Distributions on a $3\lambda$ plate at edge-on incidence}
%   \label{fig:edgeon}
% \end{figure}
%
% \begin{figure}[h]
%   \normalsize
%   \centering
%   \includestandalone[width=.5\textwidth]{figures/currents_normal_3}
%   \caption{Current Distributions on a $3\lambda$ plate at normal incidence}
%   \label{fig:normal}
% \end{figure}
%
\subsection{Far-field}
%
The scattered electric field in the far-zone can be expressed by normalizing the large argument approximation of Hankel functions as:
%
\begin{equation}
  \lim_{k_1|\v \p - \v \p'|\to\inf} E_z(\v \p) \simeq \int \limits_{0}^{L} J_z(x') e^{j k_1 x' \cos(\phi_i)} \mathrm{d}x'
  \label{eq:far-field}
\end{equation}
%
where $\phi_i$ is the angle of incidence. The results are shown in Fig. \ref{fig:rcs}.
%
\begin{figure}[h]
  \begin{center}
    \noindent
    \includegraphics[width=3in]{figures/farfield_1255.tex}
    \caption{Backscattered fields from different sheets of length $1.25\lambda$}
    \label{fig:rcs}
  \end{center}
\end{figure}
% \begin{figure}[h]
%   \centering
%   \includestandalone[width=.5\textwidth]{figures/farfield_1255}
%   \caption{Backscattered fields from different sheets of length $1.25\lambda$}
%   \label{fig:rcs}
% \end{figure}
% %
\section{Conclusion}
%
We present a new class of surface integral equations for infinitesimally thin dielectric sheets based on surface equivalence theorem. The electromagnetic response to external radiation is investigated. Results are shown for the current induced and far-zone response. Material characterization using measurable physical quantities is also outlined.
%
\bibliographystyle{IEEEtran}
% argument is your BibTeX string definitions and bibliography database(s)
\bibliography{mybib}
\end{document}
