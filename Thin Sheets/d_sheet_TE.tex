\documentclass[11pt]{article}
% Horizontal Magnetic Dipole over a lossy half-space
\usepackage[utf8]{inputenc} % Use it to include other characters than ABC
\usepackage[T1]{fontenc}
\usepackage[cmex10]{amsmath}
\usepackage{calc}
% \usepackage{systeme} % For system of equations
\usepackage{amsfonts} % to load math symbols
\usepackage{mdwmath}
\usepackage{commath}
\usepackage{mdwtab}
\usepackage{hyperref}
\usepackage{physics} % For using the oridnary derivative nomenclature
\usepackage{datetime} % Insert date and time
\usepackage[letterpaper]{geometry}
\geometry{verbose,tmargin=1.25in,bmargin=1.25in,lmargin=1.4in,rmargin=1.15in}
\usepackage[nodisplayskipstretch,doublespacing]{setspace}
\setstretch{1.5}
\usepackage{etoolbox}
%% Nicely set the spacing between equations and text
\AtBeginDocument{%
\setlength\abovedisplayskip{4pt}
\setlength\belowdisplayskip{4pt}
\setlength\abovedisplayshortskip{4pt}
\setlength\belowdisplayshortskip{4pt}
}
% \abovedisplayskip=12pt
% \belowdisplayskip=12pt
% \abovedisplayshortskip=0pt
% \belowdisplayshortskip=7pt
% \appto{\normalsize}{\zerodisplayskips}
% \appto{\small}{\zerodisplayskips}0pt
% \appto{\footnotesize}{\zerodisplayskips}
\usepackage{tocloft}
% \usepackage[rm, tiny, center, compact]{titlesec}
\usepackage{indentfirst}
\usepackage{tocvsec2}
% \usepackage[titletoc]{appendix}
% \usepackage{appendix}
% \usepackage{tamuconfig}
%
% \usepackage{rotating}
\usepackage{graphicx}
\usepackage{pgfplots}
\usepackage{tikz}
\usepackage{standalone}
\usepackage[americanresistors,americaninductors]{circuitikz}
\usepackage{tikz-dimline} % For dimensional drawing
\usetikzlibrary{positioning}
\usetikzlibrary{arrows}
\usepackage{subfig}
% The following is done to hide ugly color boxes around the links
\usepackage{xcolor}
\hypersetup{
colorlinks,
linkcolor={red!50!black},
citecolor={blue!50!black},
urlcolor={blue!80!black}
}
% pdflatex -synctex=-1
% \usepackage{mathptmx} % Times new Roman
% \usepackage{lmroman}
%
% ------------------------------- Useful Tricks Learnt
% Use ={}& to align subequations to the left

% Use = for single equations

% Use &= for split equations

% Use commath package to properly write differential operators and derivatives.

% Use \int\limits to nicely put integral limits

% For long equations, use align environment with \notag\\ as a linebreak.

% To hide section numbers, place an asterisk after the section, e.g., \section*{}

% Put comments % in between the lines in order to avoid forming a new paragraph.

% To enter special characters into Inkspace figures, use Ctrl+U and then enter       the unicode. e.g., for \times symbol, the unicode is U+0D7. So the key entry would be Ctrl+U U+0d7 and then press enter.

% Put \eqref instead or \ref to reference equations. This will automatically put parantheses around the eq. number. amsmath package required.
%
% ----------------- To compile with references use the following order in Shell"
% 1. pdflatex filename.tex
% 2. bibtex filename (no extension)
% 3. bibtex filename (no extension)
% 4. pdflatex filename.tex
% -----------------

% Personal definitions
% Operators
\renewcommand{\v}[1]{\mathbf{#1}} % vectors
\newcommand{\ti}[1]{\tilde{#1}} % spectral representation

% Symbols
\renewcommand{\O}{\omega}  % omega
\newcommand{\E}{\varepsilon}  % epsilon
\renewcommand{\u}{\mu}  % mu
\newcommand{\p}{\rho}  % rho
\newcommand{\x}{\times}  % times
\renewcommand{\inf}{\infty}  % infinity
\newcommand{\infint}{\int\limits_{-\inf}^\inf} % integral by R
\renewcommand{\del}{\nabla}  % nabla operator
\renewcommand{\^}{\hat}  % unit vector
\newcommand*\diff{\mathop{}\!\mathrm{d}} % Define differential operator



\begin{document}


\title{\textsc{Current on a planar Dieletric plate}\\}
\date{\footnote{Last Modified: \currenttime, \today.}}
\maketitle
%
A $\mathrm{TE_z}$ plane wave is incident on an infinitesimally thin planar dielectric sheet of length $L$ at an angle $\phi_i$ as shown in Fig. \ref{fig:plate}. The scattering off the sheet is obtained by applying the \emph{surface equiavalence} theorem where the dielectric sheet is replaced by equivalent surface electric and magnetic currents generating the same fields outside the sheet as the original problem and satifying boundary conditions. The currents can be expressed in terms of the fields as:
%
\begin{subequations}
  \begin{align}
    \v J_s &=  \v{\^{n}} \x \v{H} = \v{\^{z}} \mathrm{J(\v \p')},
    \label{eq:J_s}\\
    \v M_s &=  -\v{\^{n}} \x \v{E} = \v{\^{x}} \mathrm{M(\v \p')}
    \label{eq:M_s}
  \end{align}
  \label{eq:eq_currents}
\end{subequations}
%
\begin{figure}[h]
  \centering
  \includestandalone[width=.75\textwidth]{figures/plate}
  \caption{Thin Dielectric sheet with $TE_z$ polarization}
  \label{fig:plate}
\end{figure}
%
where the normal unit vector $\v{\^{n}}$ is along the postive $y$ direction and $\v \p'$ depends $x$ and $y$. To determine the surface currents, we set up two homogeneous equivalent problems; one external to the dielectric sheet and the other inside the sheet. For the external equivalent, the total electric field $\v E_1$ consists of the plane wave incident field $\v E_i$ and scattered field $\v E_1^{scat}$ due to the equivalent currents:
%
\begin{equation}
  \v E_1 = \v E_i + \v E_1^{scat}
  \label{eq:E1}
\end{equation}
%
where
%
\begin{equation}
  \v E_i = E^0 \left(\v{\^{x}} \sin \phi_i - \v{\^{y}} \cos \phi_i \right)   e^{j k_1 (x \cos \phi_i + y \sin \phi_i)}
  \label{eq:E_i}
\end{equation}
%
with $k_1$ being the propagation constant of air and $E^0$ the amplitude of the incoming plane wave. The scattered electric field due to the equivalent currents can be expressed as:
%
\begin{equation}
  \v E_1^{scat} = \left( k_0^2 + \del \del \cdot \right) \v A
  - \frac{1}{\E_1} \del \x \v F
  \label{eq:E1scat}
\end{equation}
%
where $\v A$ and $\v F$ are the magnetic and electric vector potentials respectively, given by:
%
\begin{subequations}
  \begin{align}
    \v A &=  \frac{\u}{4 \pi} \iint\limits_{S} \v J_s(\v r') \frac{ e^{-j k_0 |\v r - \v r'|}}{|\v r - \v r'|} \diff{S'},
    \label{eq:A}\\
    \v F &=  \frac{\E}{4 \pi} \iint\limits_{S} \v M_s(\v r') \frac{ e^{-j k_0 |\v r - \v r'|}}{|\v r - \v r'|} \diff{S'}.
    \label{eq:Fig}
  \end{align}
  \label{eq:potentials}
\end{subequations}
%
with $\v r$ and $\v r'$ the position vectors for the observation and source point respectively. For a structure extending to infinity in the $z$ direction, \eqref{eq:potentials} can be re-written in its two-dimensional form:
%
\begin{subequations}
  \begin{align}
    \v A &=  \frac{\u}{4 j} \int\limits_{l} \v J_s(\v \p') H_0^{(2)}(k_0 |\v \p - \v \p'|) \diff{l'},
    \label{eq:A}\\
    \v F &=  \frac{\E}{4 j} \int\limits_{l} \v M_s(\v \p') H_0^{(2)}(k_0 |\v \p - \v \p'|) \diff{l'},
    \label{eq:Fig}
  \end{align}
  \label{eq:potentials_2d}
\end{subequations}
%
where $H_0^{(2)}(k_0 |\v \p - \v \p'|)$ is the zeroth-order Hankel function of the second kind. For a horizontally oriented infinitesimally thin planar sheet, the operator inside the parentheses in \eqref{eq:E1scat} becomes:
%
\begin{equation}
  \left( k_0^2 + \del \del \cdot \right) \longrightarrow \left( k_0^2 + \pdv[2]{}{x} \right)
  \label{eq:operator}
\end{equation}
%
and the argument of the Hankel function reduces to $k_0|x - x'|$. The component of the scattered electric field tangential to sheet becomes:
%
\begin{equation}
  \begin{split}
    E_{1,x}^{scat} &= -j \O \left(k_0^2 +  \pdv[2]{}{x} \right) A_x \\
    &= -\frac{\O \u}{4} \left(k_0^2 +  \pdv[2]{}{x} \right) \int\limits_{l} J_x(x')  H_0^{(2)}(k_0 |x - x'|) \diff{l'}
  \end{split}
  \label{eq:E1sc}
\end{equation}
%
where the partial derivative along the $y$ direction is ignored bearing in mind the thickness of the sheet. In a similar way, the tangential component of the scattered magnetic field can be expressed as:
%
\begin{equation}
  \begin{split}
    \v H_{1,z}^{scat} &= -\frac{j \O} F_z \\
    &= -\frac{\O \E_1}{4} \int\limits_{l} M_z(x') H_0^{(2)}(k_0 |x - x'|) \diff{l'}
  \end{split}
  \label{eq:H1sc}
\end{equation}

For interior equivalent configuration of the original problem, the same surface currents are used with their signs reversed. With the absence of an incident wave, the tangential components of the electric and magnetics fields valid for the interior region are:
%
\begin{subequations}
  \begin{align}
    E_{2,x}^{scat} &= \frac{\O \u}{4} \left(k_1^2 +  \pdv[2]{}{x} \right) \int\limits_{l} J_x(x')  H_0^{(2)}(k_2 |x - x'|) \diff{l'}
    \label{eq:E2sc}\\
    H_{2,z}^{scat} &= \frac{\O \E_2}{4} \int\limits_{l} M_z(x') H_0^{(2)}(k_2 |x - x'|) \diff{l'}
    \label{eq:H2sc}
  \end{align}
  \label{eq:2sc}
\end{subequations}

Once the fields in the two regions are expressed in terms of equivalent currnets, we apply the boundary conditions ensuring continuity of the tangential fields at the interface thereby obtaining expressions through which the surface currents can be computed. At the interface:
%
\begin{subequations}
  \begin{align}
    \hat{\v n} \x (\v E_1 - \v E_2) ={}& \v 0
    \label{eq:BC_E}\\
    \hat{\v n} \x (\v H_1 - \v H_2) ={}& \v 0
    \label{eq:BC_H}
  \end{align}
  \label{eq:BC}
\end{subequations}
%
The component of the electric field tangential to the sheet surface is $x$ directed. We therefore, obtain a scalar equation from \eqref{eq:E_i} and \eqref{eq:BC_E}:
%
\begin{align}
  E^0 \sin \phi_i  e^{j k_1 x \cos \phi_i} &=  \frac{\O \u}{4 k_1^2}\left(k_1^2 +  \pdv[2]{}{x} \right) \int\limits_{l} J_x(x') H_0^{(2)}(k_1 |x - x'|) \diff{l'} \notag\\
  &\qquad + \frac{\O \u}{4 k_2^2}\left(k_2^2 +  \pdv[2]{}{x} \right) \int\limits_{l} J_x(x') H_0^{(2)}(k_2 |x - x'|) \diff{l'}
  \label{eq:scalarE}
\end{align}
%
For a $TE_z$ polarized incident wave, the magnetic current obtained is:
%
\begin{align}
  \frac{E^0}{\eta_1} e^{j k_1 x \cos \phi_i} &=  \frac{\O}{4} \int\limits_{l} \E_1 M_z(x') H_0^{(2)}(k_1 |x - x'|)  +  \E_2 M_z(x') H_0^{(2)}(k_2 |x - x'|) \diff{l'}
  \label{eq:scalarH}
\end{align}
%
where $\eta_1$ is the characteristic impedance of external region. Equation  \eqref{eq:scalarE} is commonly known as \emph{Pocklington Integral equation} where the differential operator written in brackets may be moved inside the integral:
%
\begin{align}
  E^0 \sin \phi_i  e^{j k_1 x \cos \phi_i}  &=  \frac{\O \u}{4 k_1^2} \int\limits_{l} J_x(x') \left(k_0^2 +  \pdv[2]{}{x} \right) H_0^{(2)}(k_1 |x - x'|) \diff{l'} \notag\\
  &\qquad + \frac{j \O}{k_2^2} \int\limits_{l} J_x(x') \left(k_2^2 +  \pdv[2]{}{x} \right) H_0^{(2)}(k_2 |x - x'|) \diff{l'}
  \label{eq:scalarE_pock}
\end{align}
%
The second order derivative on the Hankel functions can be removed by using the recurrence relations \cite[p. 361]{}.
%
\begin{subequations}
  \begin{align}
    \dv{H_0^{(2)}(x)}{x} &= -H_{1}^{(2)}(x) + \frac{1}{x} H_0^{(2)}(x)
    \label{eq:Hankel_ID1}\\
    H_{1}^{(2)}(x)  &= \frac{x}{2} \left[H_{0}^{(2)}(x) + H_{2}^{(2)}(x)\right]
    \label{eq:Hankel_ID2}
  \end{align}
  \label{eq:Hankel_ID}
\end{subequations}
%
Moreover, a Hankel function with an argument $ k \zeta = k |x - x'|$ may be differentiated by the chain-rule:
%
\begin{equation}
  \begin{split}
    \pdv{H_0^{(2)}(k \zeta)}{x} &= \dv{H_0^{(2)}(k \zeta)}{k \zeta} \pdv{k \zeta)}{x} \\
    &= \dv{H_0^{(2)}(k \zeta)}{k \zeta} \x \frac{k (x - x')}{\zeta}
  \end{split}
  \label{eq:Han}
\end{equation}
%
By differentiating \eqref{eq:Han} one more time, we obtain:
%
\begin{align}
  \pdv[2]{H_0^{(2)}(k \zeta)}{x} &= \frac{k}{\zeta} \left[H_2^{(2)}(k \zeta) \frac{k (x - x')^2}{\zeta} - H_1^{(2)}(k \zeta) \right]
  \label{eq:difHan}
\end{align}
%
The differential operator in \eqref{eq:scalarE_pock} can now be rewritten as:
%
\begin{equation}
  \begin{split}
    \left(k_i^2 + \pdv[2]{}{x} \right) H_0^{(2)}(k_i \zeta) &= \frac{k_i^2}{2} H_0^{(2)}(k_i \zeta) + k_i^2 \left[ \frac{(x-x')^2}{\zeta^2} - \frac{1}{2} \right] H_2^{(2)}(k_i \zeta) \\
    &= \frac{k_i^2}{2} H_0^{(2)}(k_i \zeta) + k_i^2 \left( \cos^2 \psi - \frac{1}{2} \right) H_2^{(2)}(k_i \zeta) \\
    &= \frac{k_i^2}{2} H_0^{(2)}(k_i \zeta) + k_i^2 \cos {(2\psi)} H_2^{(2)}(k_i \zeta)
  \end{split}
  \label{eq:Hankel_final}
\end{equation}
%
where $i = 1,2$ and $ \cos \psi = {(x-x')/\zeta}$. The electric current \eqref{eq:scalarE_pock} can therefore be expressed as:
%
\begin{align}
  E^0 \sin \phi_i  e^{j k_1 x \cos \phi_i} &=  \frac{-j \O}{2} \int\limits_{l} J_x(x') \left[ H_0^{(2)}(k_1 \zeta) + \cos {(2 \psi)} H_2^{(2)}(k_1 \zeta) \right. \notag\\
  &\qquad \left. {} + H_0^{(2)}(k_2 \zeta) + \cos {(2\psi)} H_2^{(2)}(k_2 \zeta)\right]\diff{l'}
  \label{eq:J_final}
\end{align}
%
For a flat sheet lying along the $x$ axis, the factor $\cos \psi$ equals $1$.
%%%%%%%%%%%%%%%%%%%%%%%
%%%%%%%%%%%%%%%%%%%%%%%
%%%%%%%%%%%%%%%%%%%%%%%
% \section*{Flat Dieletric Plate}
%
%
% \begin{equation}
%   H_i^{tan} = \frac{-E_i^0}{\eta} \sin \phi_i e^{-j k_0 \cos \phi_i}
%   \label{eq:H_itan}
% \end{equation}
%
\section*{Solution with Method of Moments}

The Method of Moments (MoM) is a procedure to reduce the integral equations, like \eqref{eq:E_final} and \eqref{eq:scalarH} of the form:
%
\begin{equation}
  \mathrm L (f) = g
  \label{eq:MoM}
\end{equation}
%
where $\mathrm L$ is the integral operator $\int G(x,x') \diff{x'}$, $f$ is the unknown function and $g$ is an excitation source,into a system of linear equations in terms of the unknown function which can be approximated by a finite series of weighted subdomain expansion functions:
\begin{equation}
  f(x') \approx \sum\limits_{n = 1}^N \alpha_n f_n(x'),
  \label{eq:MOM_f}
\end{equation}
%
where $\alpha_n$'s are the unknown weighting coefficients. A common choice of  the expansion functions $f_n(x')$ is a set of orthogonal pulse functions which results in a staircase approximation of the original function:
%
\begin{equation}
  f_n(x') =
  \begin{cases}
    1, & \text{for}\; x' \text{in}\; \Delta x_n'  \\
    0, & \text{otherwise}
  \end{cases}
  \label{eq:pulse}
\end{equation}
%
Introducing a series approximation of the unknown results in a residual:
\begin{equation}
  R_N = \sum \limits_{n = 1}^N \alpha_n \mathrm L [f_n(x')] - g
  \label{eq:residual}
\end{equation}
%
In order to minimize the approximation error, a \emph{moment} or an inner product is applied on a suitable choice of testing functions $w_m$ introduced:
%
\begin{equation}
  \sum \limits_{n = 1}^N \alpha_n \langle w_m, \mathrm L [f_n(x')] \rangle - \langle w_m, g \rangle = 0, \text{for}\ m = 1,2,....,N.
  \label{eq:moment}
\end{equation}
%
where $\langle \cdot, \cdot \rangle$ represents the inner product operation which is difficult to compute. To circumvent this, delta functions are used as the testing functions and this process is called \emph{ Point Matching method}. With:
%
\begin{equation}
  w_m = \delta(x' - x_m)
  \label{eq:delta}
\end{equation}
%
equation \eqref{eq:moment} becomes:
%
\begin{equation}
  \sum \limits_{n = 1}^N \alpha_n  \mathrm L [f_n(x_m)]  = g(x_m), \text{for}\ m = 1,2,....,N.
  \label{eq:moment}
\end{equation}
%
The equation set in matrix form can be written as:
%
\begin{equation}
  \mathrm Z[\mathrm \alpha_n = \mathrm g_m
  \label{eq:matrixMoM}
\end{equation}
%
where,
\begin{equation}
  \mathrm Z =
  \begin{bmatrix}
    \mathrm L [f_1(x_1)] & \mathrm L [f_2(x_1)] & \cdots & \mathrm L [f_N(x_1)] \\
    \mathrm L [f_1(x_2)] & \mathrm L [f_2(x_2)] & \cdots & \mathrm L [f_N(x_2)] \\
    \vdots & \vdots & \ddots & \vdots \\
    \mathrm L [f_1(x_N)] & \mathrm L [f_2(x_N)] & \cdots & \mathrm L [f_N(x_N)] \\
  \end{bmatrix}
  \label{eq:Zmat}
\end{equation}
%
and the vectors $\mathrm \alpha_n$ and $\mathrm g_m$ contain the unknown coefficients and the source respectively. Once $\mathrm \alpha_n$ is found through matrix inversion in \eqref{eq:matrixMoM}, the unknown function $f$ is computed through \eqref{eq:MOM_f}.

The matrix terms for electric and magnetic currents for a dielectric sheet follow from \eqref{eq:J_final} and \eqref{eq:scalarH} respectively:
%
\begin{subequations}
  \begin{align}
    Z_{mn}^J &= \frac{\O \u}{4} \int \limits_{x_n}^{x_{n+1}}  \left[ H_0^{(2)}(k_1 |x_m - x_n|) +  H_0^{(2)}(k_2 |x_m - x_n|) \right]\diff{x'}
    \label{eq:Z_J}\\
    Z_{mn}^M &= \frac{\O}{8} \int \limits_{x_n}^{x_{n+1}}  \left[ \E_1 H_0^{(2)}(k_1 |x_m - x_n|) +  \E_2 H_0^{(2)}(k_2 |x_m - x_n|) + \E_1 H_2^{(2)}(k_2 |x_m - x_n|) + \E_2 H_2^{(2)}(k_2 |x_m - x_n|) \right]\diff{x'}
    \label{eq:Z_M}
  \end{align}
  \label{eq:Z}
\end{subequations}
%
The the integrals in the $Z_{mn}$ terms can only be evaluated with any quadrature routine except for the self terms ($m = n$) for which small argument approximations of the Hankel function are used to estimate the integral. For the self terms:
%
\begin{subequations}
  \begin{align}
    Z_{nn}^J &=  -\frac{\O \u \Delta}{8} \left\{
    2 -\frac{j}{\pi} \left[ 6 + 2 \ln \left(\frac{\mathrm e^{\gamma} k_1 \Delta}{4 \mathrm e}\right) + \frac{16}{(k_1 \Delta)^2} + 2 \ln \left(\frac{\mathrm e^{\gamma} k_2 \Delta}{4 \mathrm e}\right) + \frac{16}{(k_1 \Delta)^2} \right]   \right\}
    \label{eq:Z_J}\\
    Z_{nn}^M &= -\frac{\O \u \Delta}{4} \left\{ 2 -\frac{2 j}{\pi}\ln \left(\frac{\mathrm e^{\gamma} k_1 \Delta}{4 \mathrm e}\right) -
    \frac{2 j}{\pi}\ln \left(\frac{\mathrm e^{\gamma} k_2 \Delta}{4 \mathrm e}
    \right)
    \right\}
    \label{eq:Z_M}
  \end{align}
  \label{eq:Z}
\end{subequations}





\end{document}
